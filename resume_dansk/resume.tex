\documentclass[11pt]{article}

\usepackage[utf8]{inputenc}
\usepackage[danish]{babel}
\usepackage[T1]{fontenc}
\usepackage{float}
\usepackage{fancyhdr}
\usepackage{amsmath}
\usepackage{color}
\usepackage[export]{adjustbox}
\usepackage{graphicx}
\usepackage{lastpage}
\usepackage{enumitem}
\usepackage{wrapfig}
\usepackage[a4paper, top = 1in, bottom = 1in, left=1in,right=1in]{geometry}
\usepackage[hidelinks]{hyperref}

\hfuzz=500pt

\newcommand{\tabitem}{~~\llap{\textbullet}~~}

\newcommand{\resumeSubheading}[4]{
  \noindent\begin{tabular*}{0.98\textwidth}[t]{l@{\extracolsep{\fill}}r}
    \noindent \textbf{#3} & \textit{\small #2} \\ \vspace{-3pt} 
    \noindent \textit{\small #1} & \textit{\small #4} 
  \end{tabular*}\vspace{7pt}
}

\newcommand{\listitem}[2]{
  {\small{\tabitem{#1}}} & {\small\tabitem{#2}}\\
}

\begin{document}
\begin{center}
  \textbf{\huge{\scshape{Peter Heilbo Ratgen}}}\\ 
  \vspace{0.2cm}
  \small \href{tel:+4531330916}{31330916} $|$
  \href{mailto:peter@pratgen.dk}{\underline{peter@pratgen.dk}} $|$
  \href{https://github.com/ratgen }{\underline{github}} $|$
  \href{https://pratgen.dk}{\underline{pratgen.dk}} $|$
  \href{https://www.linkedin.com/in/peter-ratgen-a1236529/}{\underline{linkedin}}
\end{center}

\noindent\large{\scshape{Om mig}} \newline
\noindent{\rule[0.3cm]{\textwidth}{0.4pt}}

\begin{wrapfigure}{R}{0.25\textwidth}
  \vspace{-0.7cm}
  \includegraphics[width=0.24\textwidth, right]{./okay.jpg}
\end{wrapfigure}
\normalsize Jeg er en 24-årig Software Engineering studerende, samt tidligere
Datalogi studerende, som bor i Odense. Til dagligt bruger jeg det meste af tiden
på at studere, eller at arbejde på egne software projekter. Derudover bruger jeg
min fritid på at være sammen med mine venner og kæreste. Jeg har en stor
interesse for app-udvikling og frontend, samt at arbejde med Linux og alt andet
der involverer programmering og software engineering. Det som driver mig mest,
er muligheden for at lære nye færdigheder, samt at øge min produktivitet gennem
nye og spændende værktøjer.

\vspace{0.3cm}
\noindent\large{\scshape{Job}} \newline
\noindent{\rule[0.3cm]{\textwidth}{0.4pt}}

\resumeSubheading{Syddansk Universitet}{}{Instruktor}{Feb. 2022 -- Jun. 2023}\\
\indent{\small Instruktor i kurset Komponent-baseret Software Engineering for
forår 22 og 23. Instruktor i kurset Software Maintenance for efterår 22.}

\vspace{0.3cm}
\resumeSubheading{Syddansk Universitet}{}{Studenterprogrammør}{Nov. 2022 -- Feb. 2023}\\
\indent{\small Studenterprogrammør på SMARTGREEN projektet.}

\vspace{0.3cm}
\resumeSubheading{Syddansk Universitet}{}{Studenterprogrammør}{Feb. 2022 -- Okt. 2022}\\
\indent{\small Studenterprgrammør på et I4.0 projekt, hovedsageligt bidraget med
DevOps, deployment til AWS og andet relateret til deployment. Ud over dette har
jeg bidraget til en React-baseret PWA.}




\vspace{0.3cm}
\noindent\large{\scshape{Uddannelse}} \newline
\noindent{\rule[0.3cm]{\textwidth}{0.4pt}}

\resumeSubheading{Syddansk Universitet}{}{Software Engineering, kandidat}{Sep.
2022 -- Jun. 2024}\\\vspace{0.25cm}
  \vspace{-0.3cm}
  {\indent\small 
  For nuværende følger jeg disse kurser: 
  \begin{itemize}
  \setlength{\itemsep}{-1pt}
    \item Embedded Linux
    \item Statistical Machine Learning
    \item Semester Project in Trustworthy Systems
    \item Software System Analysis and Verification
    \item Model-Driven Software Development
    \item Software Technology for Internet of Things
  \end{itemize}} 

\vspace{0.3cm}

\resumeSubheading{Syddansk Universitet}{}{Software Engineering, bachelor}{Sep.
2020 -- Jun. 2022}\\\vspace{0.25cm}
{\indent\small Fuldført med et 10-tal i bacheloropgaven.}

\resumeSubheading{Syddansk Universitet}{}{Datalogi}{Sep. 2017
-- Mar. 2020}\\\vspace{0.25cm} 
{\indent\small Fuldført 130 ECTS som en del af en bachelorgrad i datalogi.}
  
\vspace{0.3cm}

\resumeSubheading{Odense Tekniske Gymnasium}{}{Højere Teknisk Eksamen (HTX)}{Aug. 2014 -- Jun. 2017}
{\small \begin{itemize}\vspace{-0.25cm}
  \setlength{\itemsep}{-1pt}
  \item Studieretning Kommunikation/IT A, Design B
    \subitem Arbejdet med grafisk design, og kommunikation.
  \item Teknikfag: Teknikfag A: Design og produktion, el A
    \subitem\footnotesize Arbejdet med PIC microcontrolleren.
\end{itemize}
} \vspace{0.5cm}

\noindent\large{\scshape{Færdigheder}} \newline
\noindent{\rule[0.3cm]{\textwidth}{0.4pt}}


  \noindent\begin{tabular*}{0.62\paperwidth}[t]{l@{\extracolsep{\fill}}l}
    \textbf{Programmeringssprog} & \textbf{Værktøjer} \\ 
    \listitem{Python}{Vim}
    \listitem{Java}{Git}
    \listitem{JavaScript}{Node.js, Express}
    \listitem{SQL}{Docker}
                       & \small{\tabitem{HTML/CSS}} \\
                       & \small{\tabitem{LaTeX}} \\
                      & \\
    \textbf{Andre værktøjer} & \textbf{Sprog}  \\
    \small{\tabitem{Adobe Photoshop}} & \small{\tabitem{Dansk, primær}} \\
    \small{\tabitem{Adobe Illustrator}} & \small{\tabitem{Engelsk}}\\
    \small{\tabitem{Adobe InDesign}} & \small{\indent Skrevet og talt} \\

  \end{tabular*}
  \vspace{7pt}

\vspace{0.5cm}

\noindent\large{\scshape{Tidligere job}} \newline
\noindent{\rule[0.3cm]{\textwidth}{0.4pt}}
\resumeSubheading{Boogies}{}{Rengøring}{Sep. 2018 -- Mar. 2022}
\vspace{0.3cm}

\resumeSubheading{Design Danmark}{}{Grafisk Designer}{Feb. 2016 -- Maj 2017}\\
\indent{\small Kreeret design af menu til en café i Odense, implementeret i
Adobe InDesign. Kreeret logoforslag til forskellige klienter i Adobe
Illustrator}
\vspace{0.3cm}

\resumeSubheading{Dreamsprit Photography}{}{Redigering/Medejer}{Mar. 2015 --
Jun. 2017}\\
\indent{\small Fritidsprojekt, hvor jeg har redigeret billeder i Adobe
Photoshop, samt forskelligt grafisk arbejde.}

\end{document}
